\section{Data Sources}\label{section:data}

To study how well LLMs can cluster English and cross-lingual news articles, we construct datasets using NewsStand~\cite{Teit08}, a system designed to allow users to read the news using a map interface. 
The system ingests articles from thousands of RSS feeds within minutes of publication and presents them to users on a map, with each article's location inferred from its geographic references. 
%This allows users to ask questions like "Where did X story happen?" (feature-based queries) and "What is going on at location Y?" (location-based queries) \cite{Teit08}. 
The NewsStand interface is dynamic, so articles are collected, processed, and clustered in real time as they are published. 
%The interface also allows users to pan and zoom in to narrow the focus to local news stories or zoom out to view globally relevant international news stories. 
% After assigning a location to each article, NewsStand aggregates the articles into clusters based on the textual content and locations referenced within the document. 
% Critically, this enables articles to be ranked by story significance and displayed to users based on the map position and zoom level selected through the interface.

Using NewsStand, we construct two datasets: \emph{NewsStandEN} and \emph{NewsStandMULTI}, containing English and non-English news articles, respectively.

\subsection{Data Pre-processing}
Both datasets are processed in the following manner.
We geo-tag each article to identify geographic terms in the article text.
Geo-tagging consists of three steps: Entity Feature Extraction, Gazetteer Record Assignment, and Geographic Name Disambiguation~\cite{Teit08}. 
The first stage, Entity Feature Extraction, involves identifying important entities in the text and collecting them in an entity feature vector (EFV). 
This is accomplished using a combination of Part-Of-Speech (POS) tagging and statistical Named-Entity Recognition (NER) tagging \cite{NER}. 
For more details on this process, see \cite{Same09d, Lieb07, Ho12}. 
Once extracted, the EFV contains words belonging to proper noun classes, including location. 
Location entities in the EFV are then assigned a set of matching locations during the Gazetteer Record Assignment stage and the toponyms are resolved during the Geographic Name Disambiguation phase \cite{Leid11, Lieb10b, Lieb11, Lieb12, Same14b, Schn21}.
The resolved toponyms are then used to determine the geo-coordinates associated with each news article.
Along with the geo-coordinates, the timestamp representing the time of publication of the article is retained to aid with clustering. In addition, the name of the news organization which published the articles, the title of the article, a description of the article (in the language the article is published in), and cleaned text of the article is stored.

\nrscomment{describe other attributes in the dataset, like uuid, title, etc.}

\nrscomment{@Avik is there a paper citation we can use or a url we can footnote for Lingua? and what is the underlying model architecture Lingua uses?}
We used Lingua, a language detection library for Python, to identify the language of the text in each article and assign an appropriate tag.
The language tags were then used to split the articles into two datasets: \emph{NewsStandEN} and \emph{NewsStandMULTI}, containing English and non-English news articles, respectively.



\subsection{NewsStandMULTI}
\nrscomment{@Avik is the data on git somewhere that we can link?}
The \emph{NewsStandMULTI}~\footnote{\url{........................}} dataset consists of 68 unlabeled news articles in 7 different languages spanning a variety of topics, including entertainment, sports, finance, and more.

Table \ref{table:newsstandlarge-stats} summarizes the language distribution of the \emph{NewsStandMULTI} dataset.

\begin{table}[ht!]
\centering
\begin{tabular}{ p{2.5cm} p{2cm} p{1.5cm}  }
 %\hline
 \multicolumn{3}{c}{\textbf{Language Distribution of \emph{NewsStandMULTI} Dataset}} \\
 \hline
 Original Language & ISO ALPHA 2 Code & Number of Articles\\
 \hline
 German             & DEU     & 47\\
 Spanish            & SPA     & 6\\
 Italian            & ITA     & 6\\
 French             & FRA     & 4\\
 Portuguese         & POR     & 2\\
 Dutch              & NLD     & 2\\
 Czech              & CES     & 1
 %\hline
\end{tabular}
\caption{Language distribution of articles in the \emph{NewsStandMULTI} Dataset. The articles sampled were published between June 2019 and April 2020}
\label{table:newsstandlarge-stats}
\end{table}

\nrscomment{@Avik should we add a column for language family to show we've covered a few language families,s o we can later say it doesnt just group to language family it actually groups to individual language meaning its totally incapable of cross-lingual clustering?}






\subsection{NewsStandEN}
\nrscomment{@Avik is the data on git somewhere that we can link?}
The NewsStand dataset containing English-only articles with ground truth cluster labels (termed \emph{NewsStandEN}~\footnote{\url{...........}}) contains a sample of 100 news articles and their metadata.
The pre-processing steps for this dataset include an additional hand labeling phase.
Cluster labels for this dataset were hand-annotated, following the method in \citeauthor{Zhang2023}~\cite{Zhang2023}.
First, one or more keywords describing the news event were assigned to each article by one of three annotators.
Then, clusters were initially constructed using the keywords, geo-locations, and timestamps for context.
The clusters were then incrementally improved by reviewing the article texts that were grouped in each cluster to determine if the cluster should be split (articles describe different but similar news events) or combined (multiple clusters contain articles about the same news event).
After several rounds of incremental cluster adjustments, the cluster assignments were finalized upon annotator agreement.